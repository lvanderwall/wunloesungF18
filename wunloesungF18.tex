%This variable contains the path leading to the "LaTeX-Def"-files.
%
%Notice: Place the "LaTeX-Def"-folder on the same harddisk as your LaTeX-files!
%Be sure not to forget the slash at the beginning and the end of the path!
%\newcommand{\defaultPath}{/Lennys_Docs/LaTeX_Def/}
\newcommand{\defaultPath}{/docs/LaTeX_Def/}

\input{\defaultPath packages}
\input{\defaultPath commands}
\input{\defaultPath preamble}
%\input{./title_page}

\allowdisplaybreaks
\reversemarginpar
\newcommand{\numberthis}[1]{%
	\stepcounter{equation}\tag{\theequation}\label{#1}%
}%

\nsection{Aufgabe 1}%
\nsubsection{Aufgabe 1a)}
\incfigs{a1a}{NWM für $t<t_S$ mit $v(t<t_S)\equiv 0\quad\rightarrow\quad$KS}{a1a}%
%
\noindent Berechne den Frequenzgang (FG) mit (formaler) KWSR als Gesamtimpedanz. Bringe nach \Eref{a1a_FG_Poly1} alle Terme auf den Hauptnenner und ordne sie:
\begin{align*}%
	\uH(\jw)&=\frac{\uU_j}{\uV}=\jw L_2+R_2+(\jw L_1+R_1)\|\lr{(}{%
		\jw L_3+R_3\|\frac{1}{\jw C}%
	}{)}\\[2.5mm]%
%
	&=\jw L_2+R_2+\frac{%
		(\jw L_1+R_1)\lr{(}{%
			\jw L_3+R_3\|\frac{1}{\jw C}%
		}{)}
	}{%
		(\jw L_1+R_1)+\lr{(}{%
			\jw L_3+R_3\|\frac{1}{\jw C}%
		}{)}%
	}\ceqn{%
		\cdot\frac{\jw R_3C+1}{\jw R_3C+1}%
	}\\[2.5mm]%
%
	&=\numberthis{a1a_FG_Poly1}\jw L_2+R_2+\blue{\frac{%
		(\jw L_1+R_1)\lr{(}{%
			(\jw)^2R_3CL_3+\jw L_3+R_3%
		}{)}%
	}{%
		(\jw L_1+R_1)(\jw R_3C+1)+\lr{(}{%
			(\jw)^2R_3CL_3+\jw L_3+R_3%
		}{)}%
	}}\\[2.5mm]%
%
	&=:\frac{%
		b_3(\jw)^3+b_2(\jw)^2+b_1(\jw)+b_0%
	}{%
		(\jw)^2\ubcom{=:a_2}{R_3C(L_1+L_2)}+(\jw)\ubcom{=:a_1>0}{(R_1R_3C+L_1+L_2)}+\ubcom{=:a_0>0}{R_1+R_3}%
	}=:\frac{\uP(\jw)}{\uQ(\jw)}%
\end{align*}
%
Definiere die Koeffizienten für den Zähler:
\begin{align*}
	b_3&:=R_3C((L_1+L_3)L_2+L_1L_3),\quad b_2:=R_3C(R_1(L_2+L_3)+R_2(L_1+L_3))+L_2(L_1+L_3)+L_1L_3>0\\[2.5mm]%
%
	b_1&:=R_2(R_1R_3C+L_1+L_3)+(R_1+R_3)L_2+R_3L_1+R_1L_3>0,\quad b_0:=(R_1+R_3)R_2+R_1R_3>0%
\end{align*}%
%
%
\nsubsection{Aufgabe 1b)}%
Sei $t<t_S$ und $\jw\rightarrow\us\im[C]$\marginpar{\fbox{$C>0$}}. Unterscheide die Fälle $C>0$ und $C=0$, beginne mit $C>0$. Prüfe $\uH(\us)$ auf Teilerfremdheit. Verwende dabei, dass Teilerfremdheit bei Polynomdivision (Poly.-Div.) erhalten bleibt, d.h. aus \Eref{a1a_FG_Poly1} folgt: \glqq$\uH(\us)$ teilerfremd\grqq $\Leftrightarrow$ \glqq$\blue{%
	\uH_1(\us)=:\frac{\uP_1(\us)}{\uQ(\us)}%
}$ teilerfremd\grqq:
%
\begin{align*}%
	\uP_1(\us)&\overset{!}{=}0\quad\arr\quad\us_1=-\frac{R_1}{L_1}\quad\vee\quad\uP_2(\us):=\us^2R_3CL_4+\us L_3+R_3\overset{!}{=}0,\\[2.5mm]%
%
	\numberthis{a1b_FG_Poly2}\uQ\lr{(}{%
		-\frac{R_1}{L_1}%
	}{)}&=0+\frac{R_1^2R_3CL_3}{L_1^2}-\frac{R_1L_3}{L_1}+R_3=\ubcom{>0}{R_3}\ubcom{\neq0\text{ (Aufgabe)}}{%
		\lr{(}{%
			1-\lr{(}{%
				-\frac{R_1^2CL_3}{L_1^2}+\frac{R_1L_3}{R_3L_1}%
			}{)}%
		}{)}%
	}\neq0%
\end{align*}%
%
Bleibt zu zeigen, dass $\uP_2(\us),\:\uQ(\us)$ teilerfremd sind. Erhalte per Poly.-Div:
%
\[%
	\frac{\uQ(\us)}{\uP_2(\us)}=1+\frac{%
		(\us L_1+R_1)(\us R_3C+1)%
	}{%
		\uP_2(\us)%
	}=:\frac{\uP_3(\us)}{\uP_2(\us)}%
\]%
%
Verwende wieder, dass Teilerfremdheit bei Poly.-Div. erhalten bleibt, d.h. \glqq$\frac{\uQ(\us)}{\uP_2(\us)}$ teilerfremd\grqq $\Leftrightarrow$ \glqq$\frac{\uP_3(\us)}{\uP_2(\us)}$ teilerfremd\grqq. Prüfe die Teilerfremdheit von $\uP_2(\us),\:\uP_3(\us)$:
%
\begin{align*}%
	\uP_3(\us)&\overset{!}{=}0\quad\arr\quad\us_2=-\frac{R_1}{L_1},&\us_3&=-\frac{1}{R_3C}\\[2.5mm]%
%
	\numberthis{a1b_teilerfremd}\uP_2\lr{(}{%
		-\frac{R_1}{L_1}%
	}{)}&=\frac{R_1^2R_3CL_3}{L_1^2}-\frac{R_1L_3}{L_1}+R_3\utcom{\Eref{a1b_FG_Poly2}}{\neq}0,&\uP_2\lr{(}{%
		-\frac{1}{R_3C}%
	}{)}&=\cancel{\frac{L_3}{R_3C}}-\cancel{\frac{R_3}{CL_3}}+R_3=R_3>0%
\end{align*}%
%
$\uP_2(\us),\:\uP_3(\us)$ sind teilerfremd, also sind auch $\uP(\us),\:\uQ(\us)$ teilerfremd! Das NWM besitzt für $t<t_S$
\[%
	\lr{.}{%
		\begin{aligned}%
			\bullet\quad n_A&=4\text{ diff.-bare Var.: }\vec{x}_A(t)=(i_{L_1},\:i_{L_2},\:i_{L_3},\:u_C)^T\\[2.5mm]%
		%
			\bullet\quad n_R&\geq 2\text{ zustandsred. Glg.: }\begin{aligned}%
				0&=i_{L_2}(t)+j(t)\\%
				0&=i_{L_1}(t)-i_{L_2}(t)-i_{L_3}(t)%
			\end{aligned}%
		\end{aligned}%
	}{\}}\begin{gathered}%
		2=4-2\geq n_A-n_R=n\geq\grad\uQ(\us)=2\\[2.5mm]%
		\arr\quad\uQ(\us)\text{ liefert alle $n=2$ nat. Freq.!}
	\end{gathered}%
\]%
%
Für $C>0$ ist $\uQ(\us)$ ein Hurwitzpolynom 2.Grades, also haben alle nat. Freq. einen negativen Realteil und das NWM ist für $t<t_S$ asympt. stabil. Zusätzlich ist $\grad{\uP(\us)}-\grad{\uQ(\us)}=3-2=1>0$, also ist das NWM bzgl. $j(t)$ (mind.) einmal differenzierend!

\lf\marginpar{\fbox{$C=0$}}Betrachte $C=0$, aus A1a) folgt: $b_3=a_2=0,\quad b_2,\:a_1\neq0\quad\arr\quad\grad\uP(\us)-\grad\uQ(\us)=2-1=1>0$, also ist das NWM bzgl. $j(t)$ wieder (mind.) einmal differenzierend. Vereinfache den FG:
\[%
	\uH(\us)\utcom{\Eref{a1a_FG_Poly1}}{=}\us L_2+R_2+\frac{%
		(\us L_1+R_1)(\us L_3+R_3)%
	}{%
		\us(L_1+L_3)+R_1+R_3%
	}=:\us L_2+R_2+\frac{\uP_4(\us)}{\uQ(\us)}%
\]%
%
Die Prüfung auf Teilerfremdheit kann aus dem Fall $C>0$ übernommen werden, nur dass in \Eref{a1b_teilerfremd} die Nullstelle $\us_3$ wegfällt -- $\uH(\us)$ ist auch für $C=0$ teilerfremd! Das NWM besitzt für $t<t_S$:
%%
%Teilerfremdheit bleibt bei Poly.-Div. erhalten, d.h. \glqq$\uH(\us)$ teilerfremd\grqq $\Leftrightarrow$ \glqq$\frac{\uP_4(\us)}{\uQ(\us)}$ teilerfremd\grqq:
%\begin{align*}%
%	\uQ(\us)&\overset{!}{=}0\quad\arr\quad\us_1=-\frac{R_1+R_3}{L_1+L_3}<0\\[2.5mm]%
%%
%	\uP_4(\us_1)&=\frac{%
%		\bigl(-L_1(\cancel{R_1}+R_3)+R_1(\cancel{L_1}+L_3)\bigr)\bigl(-L_3(R_1+\cancel{R_3})+R_3(L_1+\cancel{L_3})\bigr)%
%	}{%
%		(L_1+L_3)^2%
%	}=\ubcom{<0}{%
%		\frac{-L_1^2R_3^2}{(L_1+L_3)^2}%
%	}\ubcom{>0\text{ (Aufgabe für $C=0$)}}{%
%		\lr{(}{%
%			1-\frac{R_1L_3}{R_1L_3}%
%		}{)}^2%
%	}\neq0%
%\end{align*}%
%%
%Also sind $\uP(\us),\:\uQ(\us)$ teilerfremd! Das NWM besitzt für $t<t_S$ wegen $C=0$
\[%
	\lr{.}{\begin{aligned}%
		\bullet\quad n_A&=3\text{ diff.-bare Var.: }\vec{x}_A(t)=(i_{L_1},\:i_{L_2},\:i_{L_3})^T(t)\\[2.5mm]%
	%
		\bullet\quad n_R&\geq 2\text{ zustandsred. Glg. (wie für $C>0$)}%
	\end{aligned}}{\}}\begin{gathered}%
		1=3-2\geq n_A-n_R=n\geq\grad\uQ(\us)=1\\[2.5mm]%
	%
		\arr\quad\uQ(\us)\text{ liefert alle $n=1$ nat. Freq.!}
	\end{gathered}%
\]%
%
Die einzige nat. Freq. ist $\us_4=-\frac{R_1+R_3}{L_1+L_3}<0$, also ist das NWM für $t<t_S$ asympt. stabil!
%
\anm Für $C=0$ vereinfacht sich die Zweiggleichung (ZGL) der Kapazität zu $i_C(t)=0$: Damit kommt in keiner ZGL mehr die Ableitung von $u_C(t)$ vor, d.h. $u_C(t)$ ist für $C=0$ \textit{nicht} diff.-bar!
\input{\defaultPath ending}
