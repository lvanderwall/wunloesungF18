%This variable contains the path leading to the "LaTeX-Def"-files.
%
%Notice: Place the "LaTeX-Def"-folder on the same harddisk as your LaTeX-files!
%Be sure not to forget the slash at the beginning and the end of the path!
%\newcommand{\defaultPath}{/Lennys_Docs/LaTeX_Def/}
%\newcommand{\defaultPath}{/docs/LaTeX_Def/}
\newcommand{\defaultPath}{/run/media/lvdw/DATA/docs/latex_def/}

\input{\defaultPath packages}
\input{\defaultPath commands}
\input{\defaultPath preamble}
%\input{./title_page}

\allowdisplaybreaks
\reversemarginpar
\newcommand{\numberthis}[1]{%
	\stepcounter{equation}\tag{\theequation}\label{#1}%
}%

\nsection{Aufgabe 1}%
\nsubsection{Aufgabe 1a)}
\incfigs{a1a}{NWM für $t<t_S$ mit $v(t<t_S)\equiv 0\:\rightarrow\:$KS}{a1a}%
%
\noindent Berechne den Frequenzgang (FG) mit (formaler) KWSR als Gesamtimpedanz. Bringe nach \Eref{a1a_FG_Poly1} alle Terme auf den Hauptnenner und ordne sie:
\begin{align*}%
	\uH(\jw)&=\frac{\uU_j}{\uJ}=\jw L_2+R_2+(\jw L_1+R_1)\|\lr{(}{%
		\jw L_3+R_3\|\frac{1}{\jw C}%
	}{)}\\[2.5mm]%
%
	&=\jw L_2+R_2+\frac{%
		(\jw L_1+R_1)\lr{(}{%
			\jw L_3+R_3\|\frac{1}{\jw C}%
		}{)}
	}{%
		(\jw L_1+R_1)+\lr{(}{%
			\jw L_3+R_3\|\frac{1}{\jw C}%
		}{)}%
	}\ceqn{%
		\cdot\frac{\jw R_3C+1}{\jw R_3C+1}%
	}\\[2.5mm]%
%
	&=\numberthis{a1a_FG_Poly1}\jw L_2+R_2+\blue{\frac{%
		(\jw L_1+R_1)\lr{(}{%
			(\jw)^2R_3CL_3+\jw L_3+R_3%
		}{)}%
	}{%
		(\jw L_1+R_1)(\jw R_3C+1)+\lr{(}{%
			(\jw)^2R_3CL_3+\jw L_3+R_3%
		}{)}%
	}}\\[2.5mm]%
%
	&=:\frac{%
		b_3(\jw)^3+b_2(\jw)^2+b_1(\jw)+b_0%
	}{%
		(\jw)^2\ubcom{=:a_2}{R_3C(L_1+L_3)}+(\jw)\ubcom{=:a_1>0}{(R_1R_3C+L_1+L_3)}+\ubcom{=:a_0>0}{R_1+R_3}%
	}=:\frac{\uP(\jw)}{\uQ(\jw)}%
\end{align*}
%
Definiere die Koeffizienten für den Zähler:
\begin{align*}
	b_3&:=R_3C((L_1+L_3)L_2+L_1L_3),\quad b_2:=R_3C(R_1(L_2+L_3)+R_2(L_1+L_3))+L_2(L_1+L_3)+L_1L_3>0\\[2.5mm]%
%
	b_1&:=R_2(R_1R_3C+L_1+L_3)+(R_1+R_3)L_2+R_3L_1+R_1L_3>0,\quad b_0:=(R_1+R_3)R_2+R_1R_3>0%
\end{align*}%
%
%
\nsubsection{Aufgabe 1b)}%
Sei $t<t_S$ und $\jw\rightarrow\us\im[C]$\marginpar{\fbox{$C>0$}}. Unterscheide die Fälle $C>0$ und $C=0$, beginne mit $C>0$. Prüfe $\uH(\us)$ auf Teilerfremdheit. Verwende dabei, dass Teilerfremdheit bei Polynomdivision (Poly.-Div.) erhalten bleibt, d.h. aus \Eref{a1a_FG_Poly1} folgt: \glqq$\uH(\us)$ teilerfremd\grqq $\Leftrightarrow$ \glqq$\blue{%
	\uH_1(\us)=:\frac{\uP_1(\us)}{\uQ(\us)}%
}$ teilerfremd\grqq:
%
\begin{align*}%
	\uP_1(\us)&\overset{!}{=}0\quad\arr\quad\us_1=-\frac{R_1}{L_1}\quad\vee\quad\uP_2(\us):=\us^2R_3CL_3+\us L_3+R_3\overset{!}{=}0,\\[2.5mm]%
%
	\numberthis{a1b_FG_Poly2}\uQ\lr{(}{%
		-\frac{R_1}{L_1}%
	}{)}&=0+\frac{R_1^2R_3CL_3}{L_1^2}-\frac{R_1L_3}{L_1}+R_3=\ubcom{>0}{R_3}\ubcom{\neq0\text{ (Aufgabe)}}{%
		\lr{(}{%
			1-\lr{(}{%
				-\frac{R_1^2CL_3}{L_1^2}+\frac{R_1L_3}{R_3L_1}%
			}{)}%
		}{)}%
	}\neq0%
\end{align*}%
%
Bleibt zu zeigen, dass $\uP_2(\us),\:\uQ(\us)$ teilerfremd sind. Erhalte per Poly.-Div:
%
\[%
	\frac{\uQ(\us)}{\uP_2(\us)}=1+\frac{%
		(\us L_1+R_1)(\us R_3C+1)%
	}{%
		\uP_2(\us)%
	}=:1+\frac{\uP_3(\us)}{\uP_2(\us)}%
\]%
%
Verwende wieder, dass Teilerfremdheit bei Poly.-Div. erhalten bleibt, d.h. \glqq$\frac{\uQ(\us)}{\uP_2(\us)}$ teilerfremd\grqq $\Leftrightarrow$ \glqq$\frac{\uP_3(\us)}{\uP_2(\us)}$ teilerfremd\grqq. Prüfe die Teilerfremdheit von $\uP_2(\us),\:\uP_3(\us)$:
%
\begin{align*}%
	\uP_3(\us)&\overset{!}{=}0\quad\arr\quad\us_2=-\frac{R_1}{L_1},&\us_3&=-\frac{1}{R_3C}\\[2.5mm]%
%
	\numberthis{a1b_teilerfremd}\uP_2\lr{(}{%
		-\frac{R_1}{L_1}%
	}{)}&=\frac{R_1^2R_3CL_3}{L_1^2}-\frac{R_1L_3}{L_1}+R_3\utcom{\Eref{a1b_FG_Poly2}}{\neq}0,&\uP_2\lr{(}{%
		-\frac{1}{R_3C}%
	}{)}&=\cancel{\frac{L_3}{R_3C}}-\cancel{\frac{L_3}{CR_3}}+R_3=R_3>0%
\end{align*}%
%
$\uP_2(\us),\:\uP_3(\us)$ sind teilerfremd, also sind auch $\uP(\us),\:\uQ(\us)$ teilerfremd! Das NWM besitzt für $t<t_S$
\[%
	\lr{.}{%
		\begin{aligned}%
			\bullet\quad n_A&=4\text{ diff.-bare Var.: }\vec{x}_A(t)=(i_{L_1},\:i_{L_2},\:i_{L_3},\:u_C)^T(t)\\[2.5mm]%
		%
			\bullet\quad n_R&\geq 2\text{ zustandsred. Glg.: }\begin{aligned}%
				0&=i_{L_2}(t)+j(t)\\%
				0&=i_{L_1}(t)-i_{L_2}(t)-i_{L_3}(t)%
			\end{aligned}%
		\end{aligned}%
	}{\}}\begin{gathered}%
		2=4-2\geq n_A-n_R=n\geq\grad\uQ(\us)=2\\[2.5mm]%
		\arr\quad\uQ(\us)\text{ liefert alle $n=2$ nat. Freq.!}
	\end{gathered}%
\]%
%
Für $C>0$ ist $\uQ(\us)$ ein Hurwitzpolynom 2.Grades, also haben alle nat. Freq. einen negativen Realteil und das NWM ist für $t<t_S$ asympt. stabil. Zusätzlich ist $\grad{\uP(\us)}-\grad{\uQ(\us)}=3-2=1>0$, also ist das NWM bzgl. $j(t)$ (mind.) einmal differenzierend!

\lf\marginpar{\fbox{$C=0$}}Betrachte $C=0$, aus A1a) folgt: $b_3=a_2=0,\quad b_2,\:a_1\neq0\quad\arr\quad\grad\uP(\us)-\grad\uQ(\us)=2-1=1>0$, also ist das NWM bzgl. $j(t)$ wieder (mind.) einmal differenzierend. Vereinfache den FG:
\[%
	\uH(\us)\utcom{\Eref{a1a_FG_Poly1}}{=}\us L_2+R_2+\frac{%
		(\us L_1+R_1)(\us L_3+R_3)%
	}{%
		\us(L_1+L_3)+R_1+R_3%
	}=:\us L_2+R_2+\frac{\uP_4(\us)}{\uQ(\us)}%
\]%
%
Die Prüfung auf Teilerfremdheit kann aus dem Fall $C>0$ übernommen werden, nur dass in \Eref{a1b_teilerfremd} die Nullstelle $\us_3=-\frac{1}{R_3C}$ wegfällt: $\uH(\us)$ ist auch für $C=0$ teilerfremd! Das NWM besitzt für $t<t_S$:
%%
%Teilerfremdheit bleibt bei Poly.-Div. erhalten, d.h. \glqq$\uH(\us)$ teilerfremd\grqq $\Leftrightarrow$ \glqq$\frac{\uP_4(\us)}{\uQ(\us)}$ teilerfremd\grqq:
%\begin{align*}%
%	\uQ(\us)&\overset{!}{=}0\quad\arr\quad\us_1=-\frac{R_1+R_3}{L_1+L_3}<0\\[2.5mm]%
%%
%	\uP_4(\us_1)&=\frac{%
%		\bigl(-L_1(\cancel{R_1}+R_3)+R_1(\cancel{L_1}+L_3)\bigr)\bigl(-L_3(R_1+\cancel{R_3})+R_3(L_1+\cancel{L_3})\bigr)%
%	}{%
%		(L_1+L_3)^2%
%	}=\ubcom{<0}{%
%		\frac{-L_1^2R_3^2}{(L_1+L_3)^2}%
%	}\ubcom{>0\text{ (Aufgabe für $C=0$)}}{%
%		\lr{(}{%
%			1-\frac{R_1L_3}{R_1L_3}%
%		}{)}^2%
%	}\neq0%
%\end{align*}%
%%
%Also sind $\uP(\us),\:\uQ(\us)$ teilerfremd! Das NWM besitzt für $t<t_S$ wegen $C=0$
\[%
	\lr{.}{\begin{aligned}%
		\bullet\quad n_A&=3\text{ diff.-bare Var.: }\vec{x}_A(t)=(i_{L_1},\:i_{L_2},\:i_{L_3})^T(t)\\[2.5mm]%
	%
		\bullet\quad n_R&\geq 2\text{ zustandsred. Glg. (wie für $C>0$)}%
	\end{aligned}}{\}}\begin{gathered}%
		1=3-2\geq n_A-n_R=n\geq\grad\uQ(\us)=1\\[2.5mm]%
	%
		\arr\quad\uQ(\us)\text{ liefert alle $n=1$ nat. Freq.!}
	\end{gathered}%
\]%
%
Die einzige nat. Freq. ist $\us_4=-\frac{R_1+R_3}{L_1+L_3}<0$, also ist das NWM für $t<t_S$ asympt. stabil!
%
\anm Für $C=0$ vereinfacht sich die Zweiggleichung (ZGL) der Kapazität zu $i_C(t)=0$: Damit kommt in keiner ZGL mehr die Ableitung von $u_C(t)$ vor, d.h. $u_C(t)$ ist für $C=0$ \textit{nicht} diff.-bar!
%
%
\nsubsection{Aufgabe 1c)}%
Sei $t<t_S$. Vereinfache $j(t)$ mit (*) $\sin[x]\sin[y]=\frac{1}{2}\lr{(}{\cos[x-y]-\cos[x+y]}{)},\quad x,y\im$:
\[%
	j(t)\utcom{(*)}{=}\ubcom{%
		\begin{gathered}%
			\scriptstyle E_1 :=J_0+\frac{J_1}{2}\\%
			\scriptstyle \omega_1 :=0\\%
			\scriptstyle \varphi_1 :=0%
		\end{gathered}%
	}{%
		J_0+\frac{J_1}{2}%
	}\ubcom{%
		\begin{gathered}%
			\scriptstyle E_2 :=-\frac{J_1}{2}\\%
			\scriptstyle \omega_2 :=2\omega_0\\%
			\scriptstyle \varphi_2 :=-2\omega_0t_0%
		\end{gathered}%
	}{-\frac{J_1}{2}\cos[2\omega_0(t-t_0)]}=\sum_{k=1}^2E_k\cos[\omega_kt+\varphi_k]%
\]%
%
Das als asympt. stabil vorausgesetzte NWM existiert für alle Zeiten und befindet sich für $t<t_S$ im HEZ, weil $j(t)$ rein harmonisch ist. Vereinfache den FG aus A1a) mit $R_i=R,\:L_i=L$:
\[%
	\uH(\jw)=\frac{%
		(\jw)^33RCL^2+(\jw)^2L(4R^2C+3L)+(\jw)R(R^2C+6L)+3R^2%
	}{%
		(\jw)^22RCL+(\jw)(R^2C+2L)+2R%
	}%
\]%
%
Zerlege $\uH(\jw)=:K(\omega)e^{j(\varphi_P(\omega)-\varphi_Q(\omega))}$ in Betrag und Phase:
\begin{align*}%
	\text{Betrag: }K(\omega)&:=\abs{\uH(\jw)}=\sqrt{%
		\frac{%
			(\omega R(R^2C+6L)-\omega^3 3RCL^2)^2 + (3R^2-\omega^2L(4R^2C+3L))^2%
		}{%
			(\omega(R^2C+2L))^2 + (2R-\omega^2 2RCL)^2%
		}%
	},\quad K(0)=\frac{3R}{2}\\[2.5mm]%
%
	\text{Zähler: }\varphi_P(\omega)&:=\arctg[%
		\frac{%
			\omega R(R^2C+6L)-\omega^3 3RCL^2%
		}{%
			3R^2-\omega^2L(4R^2C+3L)%
		}%
	]+\case{%
		0,&3R^2-\omega^2L(4R^2C+3L)>0,\quad\omega=\omega_1\\[2.5mm]%
		\pi,&3R^2-\omega^2L(4R^2C+3L)<0,\quad\omega=\omega_2%
	}\\[2.5mm]%
%
	\text{Nenner: }\varphi_Q(\omega)&:=\arctg[%
		\frac{%
			\omega(R^2C+2L)%
		}{%
			2R-\omega^2 2RCL%
		}%
	]+\case{%
		0,&2R-\omega^2 2RCL>0,\quad\omega=\omega_1,\:\omega_2\\[2.5mm]%
		\pi,&2R-\omega^2 2RCL<0,\quad\text{hier nicht}%
	}%
\end{align*}%
%
Prüfe die Fallunterscheidungen:
\begin{align*}%
	3R^2-\omega_1^2 L(4R^2C+3L)&=3R^2>0,&3R^2-\omega_2^2 L(4R^2C+3L)&=\ubcom{>0}{3R^2}\ubcom{%
		<0\text{ (Aufgabe)}%
	}{\lr{(}{%
		1-4\omega_0^2\lr{(}{%
			\frac{4}{3}CL+\frac{L^2}{R^2}%
		}{)}%
	}{)}} <0\\[2.5mm]%
%
	2R-\omega_1^2 2RCL&=2R>0,&2R-\omega_2^2 2RCL&=\ubcom{>0}{2R}\ubcom{%
		>0\text{ (Aufgabe)}%
	}{%
		(1-4\omega_0^2CL)%
	}>0%
\end{align*}%
%
Berechne $u_j(t)$ per Superposition im HEZ:
\[%
	u_j(t)=u_{j,hez}(t)=\sum_{k=1}^2 E_k K(\omega_k)\cos[\omega_k t+\varphi_k+\varphi_P(\omega_k)-\varphi_Q(\omega_k)]%
\]%
%
%
\nsubsection{Aufgabe 1d)}%
Sei $t<t_S$. Die AWe vor dem Schalten $\vec{x}_A(t_S^-)$ erfüllen nach A1b) zwei ZRGen, die zweite liefert
\eqn{\label{a1d_ZRG_ts-}%
	i_{L_1}(t)-i_{L_2}(t)-i_{L_3}(t)=0\quad\arr\quad i_{L_1}(t_S^-)-i_{L_2}(t_S^-)-i_{L_3}(t_S^-)=0%
}%
%
Sei nun $t>t_S$. Die STQ $j(t)$ wird kurzgeschlossen und kann weggelassen werden:
\incfigs{a1d}{NWM für $t>t_S$. Ab e) ist $R_i=R,\:C=0,\:L_i=L\quad\arr\quad C\:\rightarrow\:\text{LL}$}{a1d}%

\noindent Das NWM besitzt wegen $C>0$
\[%
	\lr{.}{\begin{aligned}%
		\bullet\quad&\text{keine gesteuerten Quellen}\\[2.5mm]%
	%
		\bullet\quad&R,\:C,\:L\neq 0\text{ nur mit je gleichem VZ}%
	\end{aligned}}{\}}\quad\arr\quad\begin{minipage}[c]{6cm}%
		Es kann nur ZRGen vom Typ \glqq\casetxt{MGL}{SGL} nur aus \casetxt{$C$}{$L$} und/oder festen \casetxt{SPQ}{STQ}\grqq\: besitzen!%
	\end{minipage}%
\]%
%
Das NWM aus \Fref{a1d} besitzt genau $n_R=1$ zustandsreduzierende Gleichung (ZRG) dieser Typen:
\[%
	S_1:\quad i_{L_1}(t)-i_{L_2}(t)-i_{L_3}(t)=0\quad\arr\quad i_{L_1}(t_S^+)-i_{L_2}(t_S^+)-i_{L_3}(t_S^+)=0%
\]%
%
Ein Vergleich mit \Eref{a1d_ZRG_ts-} zeigt, dass die AWe $\vec{x}_A(t_S^-)$ auch alle ZRGen bei $t=t_S^+$ erfüllen -- also sind die AWe bei $t=t_S^-$ hier (trotz ZRGen nach dem Schalten) immer konsistent!
%
%
\nsubsection{Aufgabe 1e)}%
Sei $t>t_S$. Berechne die ÜF mit $R_i=R,\:C=0,\:L_i=L$ im NWM aus \Fref{a1d} per doppeltem Spannungsteiler in Impedanzen. Verwende außerdem (*) $\uZ(\us):=\us L+R$:
\begin{align*}%
	\uH(\us)&:=\frac{%
		\uU_{R_1}(\us)%
	}{%
		\uV(\us)%
	}=\frac{%
		\uU(\us)%
	}{%
		\uV(\us)%
	}\cdot\frac{%
		\uU_{R_1}(\us)%
	}{%
		\uU(\us)%
	}=\frac{%
		\us L_1+R_1%
	}{%
		(\us L_1+R_1)+(\us L_2+R_2)\|(\us L_3+R_3)%
	}\cdot\frac{R_1}{\us L_1 + R_1}\ceqn{%
		A:=\frac{R}{L}>0%
	}\\[2.5mm]%
%
	&\utcom{(*)}{=}\frac{\uZ(\us)}{\uZ(\us)+\uZ(\us)\|\uZ(\us)}\cdot\frac{R}{\us L+R}=\frac{1}{%
		1+\frac{1}{2}%
	}\cdot\frac{R}{\us L+R}=\frac{2A}{3}\cdot\frac{1}{\us+A}\quad\multimapdotbothB\quad u_{R_1,\delta}(t)=\frac{2A}{3}\Theta(t)e^{-At}%
\end{align*}%
%
%
\nsubsection{Aufgabe 1f)}%
Sei $t<t_S$.\marginpar{\fbox{$t<t_S$}} Das als asympt. stabil vorausgesetzte NWM existiert für alle Zeiten und befindet sich für $t<t_S$ in Ruhe, da beide Quellen gleich Null sind: $j(t)=0$ für alle Zeiten, und $v(t<t_S)\equiv0$ folgt aus $t_a>t_S$. Das bedeutet, dass alle Strome und Spannungen Null sind:
\[%
	\vec{x}_A(t)=\vec{0}=\vec{x}_A(t_S^-),\qquad u_{R_1}(t)=0,\qquad t<t_S%
\]%
%
\marginpar{\fbox{$t>t_S$}}Sei $t>t_S$. Die AWe $\vec{x}_A(t_S^-)=\vec{0}$ werden nach A1d) stetig übernommen, also wird das NWM durch $v(t)$ aus dem Ruhezustand (RZ) angeregt! Berechne $u_{R_1}(t)$ als Antwort aus dem RZ per Faltung, definiere dazu die Hilfsfunktion $g_3(t)$ (Typ: \glqq nichtlinear\grqq):
\[
	\begin{aligned}%
		\uG_3(\us)&:=\frac{2!}{\us^3}\uH(\us)=\frac{4A}{3}\cdot\frac{1}{\us^3(\us+A)}\utcom{Zuhalte-M.}{=}\frac{4A}{3}\lr{(}{%
			\frac{\frac{1}{A}}{\us^3} + \frac{X}{\us^2} + \frac{Y}{\us} + \frac{-\frac{1}{A^3}}{\us + A}%
		}{)}\\[2.5mm]%
%
		&\multimapdotbothBvert\\[2.5mm]%
%
		g_3(t)&:=\lr{(}{%
			\Theta(t')t'^2*u_{R_1,\delta}(t')%
		}{)}(t)=\frac{%
			4\cancel{A}\Theta(t)%
		}{%
			3A^{\cancel{3}^2}%
		}\lr{(}{%
			\frac{(At)^2}{2}-At+1-e^{-At}%
		}{)},\\[2.5mm]%
%
		u_{R_1}(t)&=\lr{(}{%
			v(t')*_{R_1,\delta}(t')%
		}{)}(t)=V_0g_3(t-t_a),\quad t>t_S%
	\end{aligned}\ceqn{\begin{aligned}%
		&\scriptstyle\text{Koeff.-Vgl. $\us^3$:}\\%
		&\scriptstyle 0\overset{!}{=}\frac{1}{A}\cdot 0 + X\cdot 0 + Y\cdot 1 -\frac{1}{A^3}\cdot 1\\%
		&\scriptstyle \arr\quad Y=\frac{1}{A^3}\\[2.5mm]%
		\hline&\scriptstyle \text{Koeff.-Vgl. $\us^1$:}\\%
		&\scriptstyle 0\overset{!}{=}\frac{1}{A}\cdot 1+X\cdot A+Y\cdot 0-\frac{1}{A^3}\cdot 0\\%
		&\scriptstyle \arr\quad X=-\frac{1}{A^2}%
	\end{aligned}}%
\]%
%
\anm Die obige Darstellung für $u_{R_1}(t)$ gilt sogar für alle $t\im$.
%
%
\nsubsection{Aufgabe 1g)}%
Berechne zunächst $u_{R_1}(t)$ für alle Zeiten. Sei $t<t_S$.\marginpar{\fbox{$t<t_S$}} Das als asympt. stabil vorausgesetzte NWM existiert für alle Zeiten und befindet sich für $t<t_S$ im DC-EZ, da $v(t)=V_0,\: j(t)=J_0$:
\incfigs{a1g_DC}{DC-ESB für $t<t_S$ mit $R_i=R$}{a1g_DC}%

\noindent Berechne $u_{R_1}(t)$ und die AWe vor dem Schalten $\vec{x}_A(t_S^-)$ im DC-ESB per Superposition und einfachem Stromteiler in Impedanzen:
\begin{align*}%
	i_{L_1}(t)&=-\frac{V_0}{R_1+R_3}-\frac{R_3}{R_1+R_3}J_0\ucom{R_i=R}{=}-\frac{V_0+RJ_0}{2R}=:i_{L_1}(t_S^-),&i_{L_2}(t)&=-J_0=i_{L_2}(t_S^-),\\[2.5mm]%
%
	i_{L_3}(t)&=i_{L_1}(t)-i_{L_2}(t)=\frac{RJ_0-V_0}{2R}=i_{L_3}(t_S^-)&\arr\quad u_{R_1}(t)&=-R_1i_{L_1}(t)=\frac{V_0+RJ_0}{2}%
\end{align*}%
%
\marginpar{\fbox{$t\geq t_S$}}Sei nun $t\geq t_S$. Es gibt AWe $\neq 0$, aber keine Wartebedingung gegenüber $t_S$, berechne deshalb $u_{R_1}(t)$ per allgemeinen NW-Analyse: $u_{R_1}(t)=u_{R_1,A}(t)+u_{R_1,Q}(t),\quad t\geq t_S$
%
\begin{itemize}%
	\item $u_{R_1,A}(t)$ (Antwort nur auf AWe): Setze alle festen Quellen $=0$ und zeichne das Laplace-ESB:
	\incfigs{a1g_Lapl}{Laplace-ESB für $t\geq t_S$ mit $R_i=R,\:L_i=L$ (Zeitverschiebung: $\tilde{t}(t):=t-t_S$)}{a1g_Lapl}%
	
	\noindent Berechne $\tilde{\uI}_{L_1,A}(\us),\:\tilde{\uU}_{L_1,A}(\us)$  per Maschen-Z-Verfahren, verwende wieder $A:=\frac{R}{L}$:
	\begin{gather*}%
		(\us L+R)\LGSMat{rr}{%
			2&-1\\[2.5mm]%
			-1&2%
		}{}\LGSMat{c}{%
			\tilde{\uI}_{L_1,A}\\[2.5mm]%
			\tilde{\uI}_{L_2,A}%
		}{}(\us)=L\LGSMat{c}{%
			i_{L_1}(t_S^-)+i_{L_3}(t_S^-)\\[2.5mm]%
			i_{L_2}(t_S^-)-i_{L_3}(t_S^-)%
		}{}=\frac{L}{2R}\LGSMat{c}{%
			-2V_0\\[2.5mm]%
			V_0-3RJ_0%
		}{}\\[2.5mm]%
	%
		\arr\quad\tilde{\uU}_{R_1,A}(\us)=-R\tilde{\uI}_{L_1,A}(\us)\utcom{Cramer}{=}\frac{%
			-\cancel{R}L%
		}{%
			2\cancel{R}(\us L+R)%
		}\cdot\frac{%
			-4V_0+(V_0-3RJ_0)%
		}{3}=\frac{%
			\cancel{3L}(V_0+RJ_0)%
		}{%
			\cancel{6L}^2(\us+A)%
		},\\[2.5mm]%
	%
		\multimapdotbothBvert\quad\Laplinv[\tilde{t}]{}\\[2.5mm]%
	%
		\tilde{u}_{R_1,A}(\tilde{t})=\frac{V_0+RJ_0}{2}\Theta(\tilde{t})e^{-A\tilde{t}},\quad\tilde{t}\geq0\quad\arr\quad u_{R_1,A}(t)=\tilde{u}_{R_1,A}(t-t_S),\quad t\geq t_S
	\end{gather*}%
%
	\item $u_{R_1,Q}(t)$ (Antwort nur auf feste Quellen): Berechne $u_{R_1,Q}(t)$ per Faltung:
	\[%
		u_{R_1,Q}(t)=\lr{(}{%
			\Theta(t'-t_S)v(t')*u_{R_1,\delta}(t')%
		}{)}(t),\qquad v_Q(t):=\Theta(t-t_S)v(t)=V_0\Theta(t-t_S)%
	\]%
	%
	Definiere die Hilfsfunktion $g_1(t)$ mit der ÜF aus A1e):
	\begin{align*}%
		\uG_1(\us)&:=\frac{1}{\us}\uH(\us)=\frac{2A}{3}\cdot\frac{1}{\us(\us+A)}\utcom{Zuhalte}{=}\frac{2}{3}\lr{(}{%
			\frac{1}{\us}-\frac{1}{\us+A}%
		}{)}\\[2.5mm]%
	%
		g_1(t)&:=\lr{(}{%
			\Theta(t')*u_{R_1,\delta}(t')%
		}{)}(t)=\frac{2}{3}\Theta(t)\lr{(}{%
			1-e^{-At}%
		}{)}\quad\arr\quad u_{R_1,Q}(t)=V_0g_1(t-t_S),\qquad t\geq t_S%
	\end{align*}%
\end{itemize}%
%
Nun ist $u_{R_1}(t)$ für alle $t\im$ bekannt, berechne $i_{L_1}^{(1)}(t)$ über die Ableitung im Bereich von Distributionen:
\begin{align*}%
	i_{L_1}^{(1)}(t)&=-i_{R_1}^{(1)}(t)=-\frac{1}{R_1}u_{R_1}^{(1)}(t)=-\frac{1}{R}\lr{[}{%
		\frac{V_0+RJ_0}{2}\lr{(}{%
			1-\Theta(t-t_S)+\Theta(t-t_S)e^{-A(t-t_S)}%
		}{)}+V_0g_1(t-t_S)%
	}{]}^{(1)}\\[2.5mm]%
%
	&=-\frac{1}{R}\lr{[}{%
		\frac{V_0+RJ_0}{2}+\lr{(}{%
			\frac{2V_0}{3}-\frac{V_0+RJ_0}{2}%
		}{)}\Theta(t-t_S)\lr{(}{%
			1-e^{-A(t-t_S)}%
		}{)}%
	}{]}^{(1)}\ceqn{\text{Ausblendeigenschaft}}\\[2.5mm]%
%
	&=-\frac{1}{R}\cdot\frac{V_0-3RJ_0}{6}A\Theta(t-t_S)e^{-A(t-t_S)}=\frac{3RJ_0-V_0}{6L}\Theta(t-t_S)e^{-A(t-t_S)},\qquad t\im%
\end{align*}%
%
%\begin{align*}%
%	i_{L_1}^{(1)}(t)&=-i_{R_1}^{(1)}(t)=-\frac{1}{R_1}u_{R_1}^{(1)}(t)\utcom{\parbox{1.8cm}{\centering$u_{R_1}(t)$ stetig\\ in $t=t_S$}}{=}\frac{-1}{R_1}\case{%
%		t<t_S:&0\\[2.5mm]%
%		t\geq t_S:&\lr{(}{%
%			-A\frac{V_0+RJ_0}{2} -(-A) \frac{2V_0}{3}%
%		}{)}\Theta(t-t_S)e^{-A(t-t_S)}%	
%	}\\[2.5mm]%
%%
%	&=-A\frac{V_0-3RJ_0}{6R}\Theta(t-t_S)e^{-A(t-t_S)}\ucom{A=\frac{R}{L}}{=}-\frac{V_0-3RJ_0}{6L}\Theta(t-t_S)e^{-A(t-t_S)},\qquad t\im%
%\end{align*}%
%
\anm Nach dem Schalten wird $u_{R_1}(t)$ für $t\geq t_S$ berechnet (statt für $t>t_S$ laut Kap.8 des Skripts), weil in \Fref{a1g_Lapl} die AWe \textit{vor} dem Schalten $\vec{x}_A(t_S^-)$ eingezeichnet wurden statt der AWe nach dem Schalten $\vec{x}_A(t_S^+)$, wie in Kap.8. Laut Kap.10 des Skripts berechnet die Laplace-Trafo durch diesen Unterschied die schnellen Ausgleichsvorgänge bei $t=t_S$ automatisch mit.
%
\anm Bei $t=t_S$ treten in $i_{L_1}^{(1)}(t)$ keine Dirac-Anteile auf, weil $u_{R_1}(t)$ bei $t=t_S$ stetig ist und deshalb beim Ableiten dort keine Dirac-Anteile liefert.
%
%
\nsection{Aufgabe 2}%
\nsubsection{Aufgabe 2a)}%
\incfigs{a2a}{NWM im Laplace-Bereich}{a2a}%
%
\nsubsection{Aufgabe 2b), c)}%
Das NWM in \Fref{a2a} besitzt drei Zusatzströme für das modifizierte Knoten-$\uY$-Verfahren:
\[%
	\uI_1(\us)\text{ ist Steuerstrom,}\qquad \uI_{r_{1,2}}(\us)\text{ sind Ströme von Zweigen, die nur aus idealen SPQen bestehen}%
\]%
%
Stelle die Matrix mit den Regeln aus dem Skript direkt auf. Verwende $\uZ_1(\us):=R_1+\frac{1}{\us C_1},\: G_i:=\frac{1}{R_i}$:
\[%
	\LGSMat{cccc|ccc}{%
		\us C_3&		-\us C_3&	0&		0&		1&				0&	0\\[2.5mm]%
		-\us C_3&\us(C_2+C_3)&	0&		0&		0&				-1&	0\\[2.5mm]%
		0&				0&		G_3&	0&		0&				0&	1\\[2.5mm]%
		0&				0&		0&		G_2&	0&				1&	0\\[1.25mm]%
\hline&&&&&&\\[-2.5mm]%
		1&				0&		0&		0&		-\uZ_1(\us)&		0&	0\\[2.5mm]%
		0&				-1&		0&		1&		0\red{-r_1}&		0&	0\\[2.5mm]%
		0&				0&		1&		0&		0\red{-r_2}&		0&	0%
	}{}\ubcom{=:\vec{\uX}(\us)}{\LGSMat{c}{%
		\UPot_1\\[1mm]%
		\UPot_2\\[1mm]%
		\UPot_3\\[1mm]%
		\UPot_4\\[1mm]%
\hline\\[-3.5mm]%
		\uI_1\\[1mm]%
		\uI_{r_1}\\[1mm]%
		\uI_{r_2}%
	}{}(\us)}=\LGSMat{c}{%
		-\uJ(\us)+C_3u_{C_3,0}\\[2.5mm]%
		-C_3u_{C_3,0}+C_2u_{C_2,0}\\[2.5mm]%
		\uJ(\us)\\[2.5mm]%
		0\\[2.5mm]%
	\hline\\[-2.5mm]%
		\frac{u_{C_1,0}}{\us}\\[2.5mm]%
		\cancel{r_1\uI_1(\us)}\\[2.5mm]%
		\cancel{r_2\uI_1(\us)}%
	}{}%
\]%
%
%
\nsubsection{Aufgabe 2d)}%
Betrachte die Antwort aus dem RZ$\quad\arr\quad$Setze alle AW-Quellen zu Null! Verschiebe $\uJ(\us)$ in $M_1$:
\incfigs{a2d}{Verschiebe $\uJ(\us)$ in $M_1$}{a2d}%

\noindent Stelle die (Rück-)Trafo- und Rückgewinnungsgleichungen auf:
\begin{align*}%
	\text{Trafo-Glg.:}&&\LGSMat{c}{%
		\uI'_1\\[2.5mm]%
		\uI'_{r_2}%
	}{}(\us)&=\LGSMat{c}{%
		\uI_1\\[2.5mm]%
		\uI_{r_2}%
	}{}(\us)+\uJ(\us)\LGSMat{r}{%
		1\\[2.5mm]%
		-1%
	}{}(\us)\\[2.5mm]%
%
	\numberthis{a1d_Rueckg}\text{Rücktrafo-Glg.:}&& \LGSMat{c}{%
		\uI_1\\[2.5mm]%
		\uI_{r_2}%
	}{}(\us)&=\LGSMat{c}{%
		\uI'_1\\[2.5mm]%
		\uI'_{r_2}%
	}{}(\us)+\uJ(\us)\LGSMat{r}{%
		-1\\[2.5mm]%
		1%
	}{}(\us)\\[2.5mm]%
%
	\text{Rückgewinngung:}&&\uI_J(\us)&=\uJ(\us),\qquad\uU_J(\us)=\uU_1(\us)-\uU_{r_2}(\us)
\end{align*}%
%
%
\clearpage\nsubsection{Aufgabe 2e)}%
Zeichne den Graph des transformierten NWMs:
\incfigs{a2e}{Graph des transformierten NWMs}{a2e}%

\noindent Stelle die Matrix mit den Regeln aus dem Skript direkt auf:
\[%
	\ubcom{=:\MMat(\us)}{\LGSMat{ccc}{%
		R_3&	0&								0\red{-r_2}\\[2.5mm]%
		0&		R_2+\frac{1}{\us C_2}&	-R_2\red{-r_1}\\[2.5mm]%
		0&		-R_2&					\uZ_1(\us)+R_2+\frac{1}{\us C_3}\red{+r_1}%
	}{}}\LGSMat{c}{%
		\MStrom_1\\[2.5mm]%
		\MStrom_2\\[2.5mm]%
		\MStrom_3%
	}{}(\us)=\ubcom{=:\MVec_0(\us)}{\LGSMat{c}{%
		\cancel{-r_2\uI_1(\us)}\red{+r_2\uJ(\us)}\\[2.5mm]%
		\cancel{-r_1\uI_1(\us)}-\frac{u_{out,0}}{\us}\red{+r_1\uJ(\us)}\\[2.5mm]%
		\frac{u_{C_1,0}}{\us}-\uZ_1(\us)\uJ(\us)-\frac{u_{C_3,0}}{\us}+\cancel{r_1\uI_1(\us)}\red{-r_1\uJ(\us)}%
	}{}}%
\]%
%
Berücksichtige die Steuerung:
\[%
	\uI_1(\us)\utcom{\Eref{a1d_Rueckg}}{=}\uI'_1(\us)-\uJ(\us)\utcom{\Fref{a2e}}{=}-\MStrom_3(\us)-\uJ(\us)%
\]%
%
%
\nsection{Aufgabe 3}%
\nsubsection{Aufgabe 3a)}%
\incfigs{a3a}{DC-ESB, $I_G\approx 0$ bei MOS-Transistoren}{a3a}%

\noindent Die Arbeitspunkte von $M_{1,2}$ lassen sich wegen \glqq$I_{D_2}=-I_{D_1}$ unbekannt\grqq\:nicht unabhängig voneinander berechnen. Bestimme für jeden Transistor zwei Kennlinien und versuche, sie ineinander einzusetzen:
\begin{align*}%
	AG_1:&&0&=U_{D_1S_1}-U_{G_1S_1}-R_1\cancel{I_{G_1}}^{\approx0}\quad\arr\quad U_{D_1S_1}\approx U_{G_1S_1},\qquad I_{D_1}\utcom{Hinweis}{=:}f(U_{G_1S_1})\\[2.5mm]%
%
	D_1:&&I_{D_1}&=-I_{D_2}-\cancel{I_{G_1}}^{\approx0}\quad\arr\quad I_{D_2}\approx-I_{D_1}=-f(U_{G_1S_1})\\[5.0mm]%
%
	M_1:&&U_{G_2S_2}&\approx\frac{R_4}{R_3+R_4}V_{DD}=\frac{%
		\SI{100}{\kilo\ohm}%
	}{%
		\SI{400}{\kilo\ohm}+\SI{100}{\kilo\ohm}%
	}\cdot\SI{4}{V}=\SI{0.8}{\volt}\ceqn{\text{einfacher Spg.-Teiler}}\\[2.5mm]%
	M_2:&&0&=-V_{DD}-U_{D_1S_1}+R_2I_{D_2}+U_{D_2S_2}\approx-V_{DD}-U_{G_1S_1}-R_2f(U_{G_1S_1})+U_{D_2S_2}\\[2.5mm]%
%
	&&\arr\quad U_{D_2S_2}&\approx V_{DD}+U_{G_1S_1}+R_2f(U_{G_1S_1})\\[2.5mm]%
\intertext{$AG_1$ und $M_2$ liefern die Kennlinie $AG_2$ abhängig von der nichtlinearen Beziehung $f(U_{G_1S_1})$:}%
	\numberthis{a3a_AG2}AG_2:&&\LGSMat{c}{%
		U_{D_2S_2}\\[2.5mm]%
		I_{D_2}%
	}{}&\approx\LGSMat{c}{%
		V_{DD}+U_{G_1S_1}+R_2f(U_{G_1S_1})\\[2.5mm]%
		-f(U_{G_1S_1})%
	}{}=\LGSMat{c}{%
		\SI{4}{\volt}+U_{G_1S_1}+\SI{2}{\kilo\ohm}f(U_{G_1S_1})\\[2.5mm]%
		-f(U_{G_1S_1})%
	}{}%
\end{align*}%
%
Für alle Werte von $U_{G_1S_1}$ aus dem ersten Kennlinienfeld (\Fref{a3a_kennlinienfeld1}) liefert die Kennlinie $AG_1$ in die zugehörigen Werte $I_{D_1}$, trage sie in \Tref{a3a_kennlinie} ein:
\incfigs{kennlinienfeld1}{Arbeitspunkt $AP_1:\quad I_{D_1}\approx\SI{-0.345}{\milli\ampere},\quad U_{D_1S_1}\approx U_{G_1S_1}\approx\SI{-0.9}{\volt}$}{a3a_kennlinienfeld1}%

\noindent Mit Hilfe von \Eref{a3a_AG2} berechne einige Punkte von $AG_2$:

\tabs{a3a}{Wertetabelle für die Kennlinien $AG_1$ und $AG_2$}{a3a_kennlinie}%

\noindent Zeichne die letzten drei Punkte aus \Tref{a3a_kennlinie} in das zweite Kennlinienfeld \Fref{a3a_kennlinienfeld2} ein und verbinde sie zur Ausgleichskurve $AG_2$. Der Schnittpunkt mit der Kennlinie $U_{G_2S_2}=0.8V$ liefert den Arbeitspunkt $AP_2$:
\incfigs{kennlinienfeld2}{Arbeitspunkt $AP_2:\quad I_{D_2}\approx\SI{0.345}{\milli\ampere},\quad U_{D_2S_2}\approx\SI{2.42}{\volt},\quad U_{G_2S_2}\approx\SI{0.8}{\volt}$}{a3a_kennlinienfeld2}%

\noindent Im ersten Kennlinienfeld \Fref{a3a_kennlinienfeld1} ergibt der Schnittpunkt der Kennlinie $AG_1$ und der Geraden $I_{D_1}=-I_{D_2}\approx\SI{-0.345}{\milli\ampere}$ schließlich den Arbeitspunkt $AP_1$.
%
%
\nsubsection{Aufgabe 3b)}%
Lies die Kleinsignalparameter aus den Kennlinienfeldern \Fref{a3a_kennlinienfeld1} und \Fref{a3a_kennlinienfeld2} ab:
\begin{align*}%
	\frac{1}{R_T}&=\frac{%
		\partial I_{D_1}%
	}{%
		\partial U_{D_1S_1}=\partial U_{G_1S_1}%
	}(AP_1)\approx\frac{%
		\red{\Delta I_{D_1}}%
	}{%
		\red{\Delta U_{D_1S_1}}%
	}(AP_1)\approx\frac{%
		0.0-(-0.8)%
	}{%
		-0.7-(-1.15)%
	}\cdot\frac{%
		\si{\milli\ampere}%
	}{%
		\si{\volt}%
	}&\arr\quad R_T&\approx\SI{1.8}{\kilo\ohm}\\[2.5mm]%
%
	\frac{1}{r_0}&=\frac{%
		\partial I_{D_2}%
	}{%
		\partial U_{D_2S_2}%
	}(AP_2)\approx\frac{%
		\red{\Delta I_{D_2}}%
	}{%
		\red{\Delta U_{D_2S_2}}%
	}(AP_2)\approx\frac{%
		0.355-0.31%
	}{%
		3.0-0.0%
	}\cdot\frac{%
		\si{\milli\ampere}%
	}{%
		\si{\volt}%
	}&\arr\quad r_0&\approx\SI{67}{\kilo\ohm}\\[2.5mm]%
%
	g&=\frac{%
		\partial I_{D_2}%
	}{%
		\partial U_{D_2S_2}%
	}(AP_2)\approx\frac{%
		\green{\Delta I_{D_2}}%
	}{%
		\green{\Delta U_{D_2S_2}}%
	}(AP_2)\approx\frac{%
		0.5-0.22%
	}{%
		0.9-0.7%
	}\cdot\frac{%
		\si{\milli\ampere}%
	}{%
		\si{\volt}%
	}=\SI{1.4}{\milli\siemens}&&%
\end{align*}%
%
%
\nsubsection{Aufgabe 3c)}%
Zeichne das KS-ESB, setze alle DC-Quellen gleich Null und ersetze die Transistoren durch ihre KS-ESBs. Dabei fallen $G_1,\:D_1$ zusammen: $R_1$ wird kurzgeschlossen und kann weggelassen werden:
\incfigs{a3c}{KS-ESB, alle Zweiggrößen sind KS-Größen}{a3c}%
%
%
\nsubsection{Aufgabe 3d)}%
Fasse $R_3,\:R_4,\:C,\: L_3$ zur Impedanz $\uZ$ zusammen, da ihre Teilströme/-spannungen weder berechnet werden sollen noch etwas steuern:
\begin{align*}%
	\uZ&:=(\jw L_3+R_3)\|R_4\|\frac{1}{\jw C}=\frac{1}{%
		\jw C+\frac{1}{R_4}+\frac{1}{\jw L_3+R_3}%
	}\ceqn{%
		\cdot\frac{%
			R_4(\jw L_3+R_3)%
		}{%
			R_4(\jw L_3+R_3)%
		}%
	}\\[2.5mm]%
%
	&=\frac{%
		R_4(\jw L_3+R_3)%
	}{%
		(\jw R_4C+1)(\jw L_3+R_3)+R_4%
	}=\frac{%
		R_4(\jw L_3+R_3)%
	}{%
		(\jw)^2R_4CL_3+\jw(R_3R_4C+L_3)+R_3+R_4%
	}%
\end{align*}%
%
Stelle die Matrix für das Maschen-$\uZ$-Verfahrens im rechten Teilnetz auf. 
\eqn{\label{a3d_MaschenZ}%
	\LGSMat{c}{%
		\jw(L_1+(1\red{+gr_0})L_2)+r_0+R_3+R_T%
	}{}\LGSMat{c}{\uI_{D_2}}{}=\LGSMat{c}{%
		\cancel{gr_0\uU_{G_2S_2}}\red{+gr_0\uZ\uJ}%
	}{}%
}%
%
Berücksichtige die Steuerung:
\[%
	\uU_{G_2S_2}=\uZ\uJ-\jw L_2\uI_{D_2}%
\]%
%
Berechne den FG:
\[%
	\uH(\jw)=\frac{\uU_{out}}{\uJ}=-(\jw L_1+R_T)\frac{\uI_{D_2}}{\uJ}\ucom{%
		\Eref{a3d_MaschenZ}%
	}{=}-\frac{%
		gr_0\uZ(\jw L_1+R_T)%
	}{%
		\jw(L_1+(1+gr_0)L_2)+r_0+R_3+R_T%
	}%
\]%
%
%
\nsection{Aufgabe 4}%
Substituiere die Eingänge der Z-Parameter-DST $i_1(t),\:i_2(t)$ durch feste STQen, nach Aufgabe darf vorausgesetzt werden, dass das NWM nach Substitution eindeutig lösbar ist. Wandle $v(t),\: R_1$ in eine ESTQ um und fasse $R_1,\:C_1$ sowie $R_3,\:C_3$ zusammen, da ihre Teilgrößen weder berechnet werden sollen noch etwas steuern. $R_0$ liegt in Reihe zur STQ $\uI_1$ und kann weggelassen, die Rückgewinnung von $\uU_1$ steht in \Fref{a4}. Verwende $\uY_i=\jw C_i+\frac{1}{R_i},\quad G_i:=\frac{1}{R_i}$:
\incfigs{a4}{%
	Vereinfachtes ESB im Bereich der KWSR nach Subst. von $\uI_{1,2},\quad \uU_1=R_0\uI_1+\UPot_1$%
}{a4}%

\noindent Die Steuerungspannung $u(t)$ kann \textit{nicht} direkt nur durch feste Quellen berechnet werden -- berechne deshalb die Ausgänge $\uU_1,\:\uU_2$ der Z-Parameter-DST mit dem Knoten-$\uY$-Verfahren! Wandle $v(t),\: R_1$ in eine ESTQ um und stelle die Matrix mit den Regeln aus dem Skript direkt auf:
\begin{multline*}%
	\PotMat\overset{\bullet}{\vec{\uU}}:=\LGSMat{ccc}{%
		\jw C_2+\uY_1&-\uY_1&-\jw C_2\\[2.5mm]%
%
		-\uY_1\red{+g}&\uY_1+G_2+G_4&-G_2\red{-g}\\[2.5mm]%
%
		-\jw C_2\red{-g}&-G_2&\jw C_2+G_2+\uY_3\red{+g}%
	}{}\LGSMat{c}{%
		\UPot_1\\%
		\UPot_2\\%
		\UPot_3%
	}{}=\\[2.5mm]%
%
	=\LGSMat{c}{%
		\uI_1-G_1\uV\\[2.5mm]%
		\uI_2+G_1\uV\cancel{-g\uU}\\[2.5mm]%
		\cancel{g\uU}%
	}{}=:\PotVec,\qquad\text{Berücksichtige die Steuerung: }\uU=\UPot_1-\UPot_3%
\end{multline*}%
%
Berechne die Z-Parameter-DST mit der Cramer'schen Regel:
\begin{multline*}%
	\LGSMat{c}{%
		\uU_1\\[2.5mm]%
		\uU_2%
	}{}\utcom{\Fref{a4}}{=}\LGSMat{c}{%
		R_0\uI_1+\UPot_1\\[2.5mm]%
		\UPot_2%
	}{}\utcom{Cramer}{=}\LGSMat{c}{%
		R_0\uI_1\\[2.5mm]%
		0%
	}{}+\frac{1}{\det[\PotMat]}\LGSMat{c}{%
		\det[\PotMat_1^*]\\[2.5mm]%
		\det[\PotMat_2^*]%
	}{}\\[2.5mm]%
%
	\utcom{Koeff.-Vgl.}{\overset{!}{=}}\LGSMat{cc}{%
		\uZ_{11}&\uZ_{12}\\[2.5mm]%
		\uZ_{21}&\uZ_{22}%
	}{}\LGSMat{c}{%
		\uI_1\\[2.5mm]%
		\uI_2%
	}{}+\LGSMat{c}{%
		\uV_{Z,1}\\[2.5mm]%
		\uV_{Z,2}%
	}{},\qquad\PotMat_i^*=\text{\glqq$\PotMat$, substituiere Spalte $i$ durch $\PotVec$\grqq}%
\end{multline*}%
\anm Fasst man die Zweige mit $i_2(t),\:R_4$ und $R_1,\:C_1,\:v(t)$ in je eine ESPQ  um und verschiebt $i_1(t)$ in einer Masche, liefert das Maschen-$\uZ$-Verfahren eine kleinere Matrix und wäre eigentlich die bessere Wahl; insbesondere, weil die Rücktrafo/Rückgewinnung in dieser Aufgabe nicht benötigt wird. Da in der Lösung aber die Determinanten-Funktion verwendet werden darf, spielt die Größe der Matrix in dieser Aufgabe keine Rolle\ldots
\input{\defaultPath ending}
